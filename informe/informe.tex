\documentclass[12pt]{article}
\usepackage{url}
\usepackage[spanish, english]{babel}
\selectlanguage{spanish}
\usepackage[fixlanguage]{babelbib}
\selectbiblanguage{spanish}
\usepackage{url}
\usepackage[utf8]{inputenc}
\usepackage{float}
\usepackage{fullpage}
\usepackage{amsmath}
\usepackage{amssymb}
\usepackage{graphicx}


\usepackage[titletoc, title]{appendix}
\addto{\captionsspanish}{\renewcommand*{\appendixname}{Anexo}}

\bibliographystyle{bababbrv}

\title{Clasificación de Dígitos Impresos: OCR}
\author{Jorge Bahamonde\\
\small{\url{jbahamon@ug.uchile.cl}}}
\date{}

\begin{document}
\maketitle

\begin{abstract}
    El reconocimiento y clasificación de caracteres impresos es un método
    conocido para la digitalización de texto, de modo de poder realizar el
    almacenamiento, búsqueda y otros análisis de mejor forma.  El estudio y
    comparación de diferentes descriptores es necesario en un área en que
    diversas variantes son propuestas en base a intuiciones. En particular, una
    clase importante de descriptores está compuesta por aquellos basados en
    nociones de concavidad. En este trabajo se estudiaron tres descriptores
    basados en concavidad para el caso de la clasificación de dígitos impresos.
    Se exploró (de forma limitada) el espacio de parámetros del clasificador
    utilizado (KNN). Posteriormente, se analizó la extensión de los resultados
    obtenidos en el caso de agregar mecanismos de división espacial en la
    obtención de los descriptores, buscando comparar el comportamiento de éstos.
\end{abstract}

\section{Introducción}

El reconocimiento de dígitos impresos es una aplicación del reconocimiento de
patrones, siendo aplicable para el procesamiento de formularios y automatización
de digitalización de documentos. En particular, la clasificación de dígitos
impresos puede ayudar, por ejemplo, en la digitalización de documentos
relacionados con contabilidad.

Un paso importante en el proceso de clasificación es el de extracción de las
características de las imágenes a clasificar, en la forma de descriptores.
Existen múltiples técnicas para este propósito, siendo necesario el análisis y
comparación de su desempeño. En particular, existe un conjunto de técnicas
orientadas a describir imágenes en base a un modelamiento de sus propiedades de
concavidad. Estos descriptores usualmente son utilizados como un complemento
para otro tipo de descriptores \cite{conc}.

Adicionalmente, una forma común de mejorar la efectividad de un descriptor es
realizar la división espacial de la imagen a describir. Así, para cada zona
creada se genera un descriptor, siendo la concatenación de todos éstos el
descriptor global. Sin embargo, una consecuencia directa de esto es un
incremento en el tamaño del descriptor.

En este trabajo se estudiaron y compararon tres descriptores basados en
conceptos de concavidad, en el contexto del reconocimiento de dígitos impresos.
Además de explorar los parámetros utilizados al momento de la clasificación, se
analizó el comportamiento observado al introducir mecanismos de división
espacial. 

\section{Descripción del Trabajo}

En este trabajo se realizó la evaluación de un algoritmo para el reconocimiento
de piel basado en modelos estadísticos de color. 

En una primera etapa, se generó un \emph{dataset}
pequeño, de 30 imágenes, desde dos fuentes. 
Este trabajo se enfocó en la clasificación de dígitos impresos

(es decir, no
escritos a mano) utilizando imágenes binarizadas con un tamaño estándar, con
dígitos negros sobre un fondo blanco.

\subsection{Modelos de color}

Una forma de modelar una característica a detectar es mediante un modelo
estadístico sobre los colores de la imagen. Un pixel con cierto valor $rgb$ es
clasificado por uno de estos modelos como ser efectivamente piel si:

\begin{equation}
    \frac{ \mathbb{P} [ rgb|skin ] }{ \mathbb{P} [ rgb | \neg skin ] } \geq
    \Theta
\end{equation}

con $\Theta$ un umbral que puede ser descrito como función del costo de tener
falsos positivos y falsos negativos:

\begin{equation}
    \Theta = \frac{c_p \mathbb{P} [\neg skin]}{c_n \mathbb{P}[skin]}
\end{equation}

Ahora bien, las distribuciones $\mathbb{P} [ rgb|skin ]$ y $\mathbb{P} [ rgb | \neg
skin ]$ pueden ser determinadas de distintas formas. En el caso de este trabajo, se
utilizó un modelo que describe estas distribuciones como una combinación de
funciones gaussianas.

\subsection{Modelo basado en Mezcla de Gaussianas}

Una forma de modelar las distribuciones de probabilidad  $\mathbb{P} [ rgb|skin
]$ y $\mathbb{P} [ rgb | \neg skin ]$ es mediante una suma de funciones
gaussianas:

\begin{equation}
    \mathbb{P}[\mathbf{x}] = \sum\limits_{i=1}^N w_i
    \frac{1}{(2\pi)^{\frac{3}{2}} | \Sigma_i |^\frac{1}{2}} e^{-\frac{1}{2}
    (\mathbf{x} - \mathbf{\mu}_i)^{\text{T}} \Sigma_i ^{-1} (\mathbf{x} -
    \mathbf{\mu}_i)},
\end{equation}

donde los valores $\mu_i$ y $\Sigma_i$ son las medias de cada gaussiana y las
correspondientes matrices (diagonales) de covarianza. Los valores $w_i$
determinan el peso que cada gaussiana ejerce en la mezcla. $N$ define la
complejidad del modelo, (ya que un mayor valor de $N$ resulta en más gaussianas
y por tanto un mayor número de grados de libertad) siendo un parámetro a
elección.

Los autores de \cite{fef} entrenaron modelos para estimar ambas distribuciones
de probabilidad descritas.

\subsection{Dataset utilizado}

\section{Evaluación y Análisis de Resultados}

\subsection{Resultados de los Experimentos}

\subsection{Discusión}

\section{Conclusiones}

\selectlanguage{spanish}
\selectbiblanguage{spanish}

\bibliography{./informe}

\begin{appendices}

\section{Tasas de falsos positivos y verdaderos positivos obtenidas}

\begingroup
\centering
\pgfplotstabletypeset[col sep=tab,
%     columns={T,FP,TP},
     display columns/0/.style={column name=$\Theta$},
     display columns/1/.style={column name=\emph{FalsePositiveRate}},
     display columns/2/.style={column name=\emph{TruePositiveRate}},
     every head row/.style={before row=\toprule,after row=\midrule},
     every last row/.style={after row=\bottomrule},
    ]{results.csv}
\captionof{figure}{Tasas de falsos positivos y verdaderos positivos obtenidas.}\label{fig:f}
\endgroup


\end{appendices}


\end{document} 
