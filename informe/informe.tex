\documentclass[12pt]{article}
\usepackage{url}
\usepackage[spanish, english]{babel}
\selectlanguage{spanish}
\usepackage[fixlanguage]{babelbib}
\selectbiblanguage{spanish}
\usepackage{url}
\usepackage[utf8]{inputenc}
\usepackage{float}
\usepackage{fullpage}
\usepackage{amsmath}
\usepackage{graphicx}


\usepackage[titletoc, title]{appendix}
\addto{\captionsspanish}{\renewcommand*{\appendixname}{Anexo}}

\bibliographystyle{bababbrv}

\title{Clasificación de Dígitos Impresos: OCR}
\author{Jorge Bahamonde\\
\small{\url{jbahamon@ug.uchile.cl}}}
\date{}

\begin{document}
\maketitle

\begin{abstract}
    El reconocimiento y clasificación de caracteres impresos es un método
    conocido para la digitalización de texto, de modo de poder realizar el
    almacenamiento, búsqueda y otros análisis de mejor forma.  El estudio y
    comparación de diferentes descriptores es necesario en un área en que
    diversas variantes son propuestas en base a intuiciones. En particular, una
    clase importante de descriptores está compuesta por aquellos basados en
    nociones de concavidad. En este trabajo se estudiaron tres descriptores
    basados en concavidad para el caso de la clasificación de dígitos impresos.
    Se exploró (de forma limitada) el espacio de parámetros del clasificador
    utilizado (KNN). Posteriormente, se analizó la extensión de los resultados
    obtenidos en el caso de agregar mecanismos de división espacial en la
    obtención de los descriptores, buscando comparar el comportamiento de éstos.
\end{abstract}

\section{Introducción}

El reconocimiento de dígitos impresos es una aplicación del reconocimiento de
patrones, siendo aplicable para el procesamiento de formularios y automatización
de digitalización de documentos. En particular, la clasificación de dígitos
impresos puede ayudar, por ejemplo, en la digitalización de documentos
relacionados con contabilidad.

Un paso importante en el proceso de clasificación es el de extracción de las
características de las imágenes a clasificar, en la forma de descriptores.
Existen múltiples técnicas para este propósito, siendo necesario el análisis y
comparación de su desempeño. En particular, existe un conjunto de técnicas
orientadas a describir imágenes en base a un modelamiento de sus propiedades de
concavidad. Estos descriptores usualmente son utilizados como un complemento
para otro tipo de descriptores \cite{conc}.

Adicionalmente, una forma común de mejorar la efectividad de un descriptor es
realizar la división espacial de la imagen a describir. Así, para cada zona
creada se genera un descriptor, siendo la concatenación de todos éstos el
descriptor global. Sin embargo, una consecuencia directa de esto es un
incremento en el tamaño del descriptor.

En este trabajo se estudiaron y compararon tres descriptores basados en
conceptos de concavidad, en el contexto del reconocimiento de dígitos impresos.
Además de explorar los parámetros utilizados al momento de la clasificación, se
analizó el comportamiento observado al introducir mecanismos de división
espacial. 

\section{Descripción del Trabajo}

Este trabajo se enfocó en la clasificación de dígitos impresos (es decir, no
escritos a mano) utilizando imágenes binarizadas con un tamaño estándar, con
dígitos negros sobre un fondo blanco.


\subsection{Descriptores basados en concavidad}

\subsubsection{Concavidad 4-Conexa (4CC)}

Para este descriptor, cada pixel del fondo (es decir, blanco) de la imagen es
codificado utilizando 4 bits. Para ello, se realiza una búsqueda de pixeles del
dígito (pixeles negros) en las cuatro direcciones principales a partir del pixel
de fondo, similar a los posibles movimientos de una torre en el ajedrez. 

Cada dirección es representada por un bit, que se setea de acuerdo a si se
encontró o no un pixel del dígito. Al formar un histograma con los códigos
resultantes, se obtiene en un descriptor de largo 16. 

\subsubsection{Concavidad 8-Conexa (8CC)}

Este descriptor es similar al anterior; sin embargo, las direcciones utilizadas
por la búsqueda son las diagonales, de manera similar a los posibles movimientos
de un alfil en el ajedrez. De esta forma, el descriptor obtenido es también un
histograma de largo 16.

\subsubsection{Concavidad de 13 Bins (13C)}

Este descriptor, definido en \cite{bins}, puede ser visto como una modificación
de 4CC. Se realiza una búsqueda en las mismas direcciones, siendo ignorados
aquellos casos en que se encuentran pixeles del dígito en una o menos
direcciones. Adicionalmente, se ignoran aquellos pixeles que encuentran dos
pixeles negros en el mismo eje (es decir, a la izquierda y a la derecha, o
arriba y abajo), si bien esto no se explicita en la referencia original. 

Adicionalmente, en el caso en que se encuentren pixeles negros en las cuatro
direcciones, se realiza una búsqueda adicional en cuatro direcciones auxiliares,
de modo de confirmar si el pixel blanco se encuentra totalmente rodeado o no.
Esto resulta en un histograma con 13 compartimientos.

Finalmente, en el caso de este trabajo, se utilizaron también versiones de estos
algoritmos en las que previamente se divide la imagen en una grilla de $W \times
W$ celdas, siendo el descriptor de la imagen total la concatenación de los
descriptores para cada celda. Así, estas versiones resultan en descriptores de
largo $W \times W \times 16$ en el caso de 4CC y 8CC, y de largo $W \times W
\times 13$ en el caso de 13C. Estas versiones modificadas se utilizaron de forma
posterior a la exploración de parámetros realizada.

Todos estos descriptores fueron extraídos utilizando programas escritos en
Python, utilizando la biblioteca OpenCV y NumPy.

\subsection{Conjuntos de Datos}

Se utilizó un conjunto de datos compuesto por imágenes de dígitos impresos en
blanco y negro (binarizadas) de 128x128 pixeles. El conjunto de datos de
entrenamiento se compuso de 7110 imágenes, 711 para cada dígito. No se utilizó
un conjunto de validación.

Por otro lado, el conjunto de datos de prueba se compuso de 3050 imágenes, 305
para cada dígito. 

\begin{figure}[h]
    \centering
    \includegraphics{digit}
\caption{Ejemplo de las imágenes con las que se trabajó.}
\end{figure}

\subsection{Clasificación}

El clasificador utilizado sobre los descriptores antes mencionados fue KNN
(\emph{K-nearest neighbors}) utilizando dos funciones de distancia: Manhattan
(L1) y Euclidiana (L2), que pueden ser vistas como casos particulares de la
distancia de Minkowski, con $p=1$ y $p=2$, respectivamente. Para el parámetro
$K$ del algoritmo, se uti:lizaron valores de 1, 5, 10 y 20, escogiéndose
aquellos valores que produjeran los mejores resultados. Se utilizó la
implementación del algoritmo KNN incluida en la biblioteca scikit-learn de
Python.

\section{Evaluación y Análisis de Resultados}

\subsection{Resultados de los Experimentos}

Se listan a continuación los resultados obtenidos. Se muestran los resultados
correspondientes a los valores de $K$ que produjeron los mejores resultados para
cada algoritmo. Se adjunta como anexo la totalidad de los resultados obtenidos,
en la forma de matrices de confusión.

Se define C$i$ como el porcentaje de clasificaciones correctas para el dígito
$i$; además, se define CT como el porcentaje de clasificación total. La
totalidad de los porcentajes de clasificación correcta para cada experimento se
incluyen como anexo.

\begin{table}[H]
\begin{tabular}{|l|c|c|c|c|c|c|c|c|c|c|c|c|}
\hline
\textbf{Desc.}&\textbf{K}&\textbf{CT}&\textbf{C0}&\textbf{C1}&\textbf{C2}&\textbf{C3}&\textbf{C4}&\textbf{C5}&\textbf{C6}&\textbf{C7}&\textbf{C8}&\textbf{C9}\\
\hline
4CC&1&93.48&96.72&95.74&90.16&92.13&97.7&91.48&91.48&98.36&88.2&92.79\\
\hline
8CC&1&94.16&91.8&97.38&98.03&96.07&98.69&91.48&84.26&99.02&91.15&93.77\\
\hline
13C&1&97.21&97.38&96.72&95.41&98.36&97.7&96.39&99.02&99.34&93.44&98.36\\
\hline
\end{tabular}
\caption{Porcentajes de clasificaciones correctas para distancia de Manhattan}
\end{table}

\begin{table}[H]
\begin{tabular}{|l|c|c|c|c|c|c|c|c|c|c|c|c|}
\hline
\textbf{Desc.}&\textbf{K}&\textbf{CT}&\textbf{C0}&\textbf{C1}&\textbf{C2}&\textbf{C3}&\textbf{C4}&\textbf{C5}&\textbf{C6}&\textbf{C7}&\textbf{C8}&\textbf{C9}\\
\hline
4CC&1&92.72&94.75&95.41&87.87&90.82&96.39&91.48&91.8&98.69&87.54&92.46\\
\hline
8CC&1&93.51&89.84&96.07&97.38&94.75&98.03&91.15&84.26&99.02&90.49&94.1\\
\hline
13C&1&96.59&96.72&95.74&94.75&97.38&97.38&96.39&98.69&99.34&92.13&97.38\\
\hline
\end{tabular}
\caption{Porcentajes de clasificaciones correctas para distancia Euclidiana}
\end{table}

De esta forma, se obtiene que los parámetros en que mejor se comportan los
métodos (utilizando el porcentaje de clasificación total como criterio de
decisión) corresponden al uso de $K=1$ para KNN, con la distancia de Manhattan
como la más efectiva.

Con estos parámetros, se realizaron experimentos utilizando división espacial
mediante grillas de $2 \times 2$ y de $4 \times 4$ celdas, con lo que se tienen
los descriptores 4CC-4, 4CC-16, 8CC-4, 8CC-16, 13C-4 y 13C-16. Estos
descriptores tienen largos 64, 256, 64, 256, 52 y 208, respectivamente. Se
presentan a continuación sus porcentajes de clasificaciones correctas. Las
correspondientes matrices de confusión se encuentran en el anexo.

\begin{table}[H]
\begin{tabular}{|l|c|c|c|c|c|c|c|c|c|c|c|}
\hline
\textbf{Desc}&\textbf{CT}&\textbf{C0}&\textbf{C1}&\textbf{C2}&\textbf{C3}&\textbf{C4}&\textbf{C5}&\textbf{C6}&\textbf{C7}&\textbf{C8}&\textbf{C9}\\
\hline
4CC-4&98.33&99.67&97.38&98.36&97.05&98.36&99.34&98.36&99.02&98.03&97.7\\
\hline
8CC-4&97.44&98.69&98.69&98.69&94.75&99.02&97.38&94.43&99.67&96.39&96.72\\
\hline
13C-4&98.0&100.0&99.34&99.02&96.07&97.38&99.02&96.39&98.03&96.07&98.69\\
\hline
4CC-16&97.51&99.67&97.7&99.34&94.1&98.36&96.72&97.05&99.67&95.08&97.38\\
\hline
8CC-16&96.98&98.36&98.36&99.67&92.79&98.69&96.72&94.43&99.67&94.1&97.05\\
\hline
13C-16&97.18&99.67&97.38&98.36&95.08&98.36&97.7&95.41&98.69&94.1&97.05\\
\hline\end{tabular}
\caption{Porcentajes de clasificaciones correctas al usar división espacial}
\end{table}

Se obtiene, así, que la mejor tasa de reconocimiento total se tiene en el caso
de 4CC-4, seguido de cerca por 13C-4.

\subsection{Discusión}

Es interesante que consistentemente se obtengan mejores resultados cuando el
valor de K es igual a 1 (es decir, el clasificador es simplemente escoger al
vecino más cercano). Es posible que en el caso de dígitos impresos exista una
variabilidad limitada: de ser así, la inclusión de más vecinos puede significar
forzar al clasificador a considerar una variabilidad que no está presente. Una
forma de estudiar esta variabilidad podría consistir en reconocer clústers
dentro de un conjunto de imágenes que correspondan a un mismo dígito, dados sus
descriptores de concavidad. ***\emph{overfitting}

A la hora de comparar los descriptores, el descriptor 13C tuvo un mejor
desempeño (al menos un 3\% de mejora en los porcentajes de clasificaciones
correctas totales) de forma consistente. Esto muestra, entonces, que las
suposiciones y modificaciones hechas por este descriptor se concretan en la
práctica. 

De todas formas, en el caso particular de algunos dígitos, 13C muestra un
comportamiento ligeramente inferior (en particular, 8CC parece tener un mejor
comportamiento en el caso del 1 y el 2), lo que pudiera explicarse por los casos
que 13C elimina. Por otro lado, 13C supera notablemente a los otros descriptores
en el caso de dígitos como el 5 y el 6, lo que concuerda con la búsqueda que se
realiza en las direcciones auxiliares para distinguir aquellos pixeles que
realmente se encuentran encerrados.

Los descriptores 4CC y 8CC tuvieron comportamientos similares en lo global,
presentando \emph{tradeoffs} en lo que concierne a dígitos individuales. En
particular, 4CC pareciera presentar dificultades para distinguir el 2 del 5, lo
que pudiera explicarse considerando que son dígitos similares salvo el
intercambio de la mitad superior con la inferior. Por otro lado, 8CC presenta un
comportamiento inferior con dígitos como el 6, confundiéndolo con el 0. Esto
puede suceder en particular cuando la curvatura del trazo superior es
pronunciada.

Los tres descriptores presentan dificultad en el caso del 8. Esto es
comprensible al recordar que estos descriptores tratan de identificar la
concavidad de forma local: así, por ejemplo, no son capaces de identificar que
un 8 posee dos zonas internas, mientras que un 0 puede poseer sólo una: la única
forma de distinguirlos consiste en el cambio en anchura del 8. Así, cobra
relevancia el uso de las direcciones auxiliares definidas por 13C.

Finalmente, es interesante notar que estas diferencias prácticamente desaparecen
cuando se realiza la división espacial de las imágenes, haciendo que los
descriptores presenten un comportamiento mucho más parecido y estable. Esto es
consistente con lo expuesto con anterioridad, ya que los fenómenos descritos no
suceden cuando la imagen está seccionada. Por ejemplo, al aplicar una grilla
sobre un 8 se obtienen subimágenes muy diferentes de las obtenidas al hacerlo
sobre un 0. Es interesante notar que el comportamiento global de 13C es el que
menos cambia al aplicar división espacial. De esta forma, puede decirse que el
aplicar división espacial a los métodos anteriores pudiera estar corrigiendo las
mismas deficiencias que 13C.

\section{Conclusiones}

A partir del trabajo realizado y los resultados obtenidos, puede apreciarse que
si bien las técnicas y razonamientos aplicadas para desarrollar el descriptor
13C parecen ser efectivas, sus mejoras parecen ser equivalentes a las logradas
mediante la división espacial de las imágenes. En este caso, entonces, la
verdadera mejora introducida por el descriptor 13C es resultar en descriptores
más cortos con un tasa de reconocimiento similar al caso en que se utilice
división espacial.

Queda abierta la posibilidad de estudiar el comportamiento del clasificador KNN
en el caso de utilizar división espacial.

\selectlanguage{spanish}
\selectbiblanguage{spanish}

\bibliography{./informe}

\begin{appendices}

\section{Tasas de falsos positivos y verdaderos positivos obtenidas}

\begingroup
\centering
\pgfplotstabletypeset[col sep=tab,
%     columns={T,FP,TP},
     display columns/0/.style={column name=$\Theta$},
     display columns/1/.style={column name=\emph{FalsePositiveRate}},
     display columns/2/.style={column name=\emph{TruePositiveRate}},
     every head row/.style={before row=\toprule,after row=\midrule},
     every last row/.style={after row=\bottomrule},
    ]{results.csv}
\captionof{figure}{Tasas de falsos positivos y verdaderos positivos obtenidas.}\label{fig:f}
\endgroup


\end{appendices}


\end{document} 
